\documentclass[12pt,a4j]{jarticle}
\usepackage[dvips]{graphicx,color}
\usepackage{amsmath,amsthm,amssymb}
%\topmargin -15mm\oddsidemargin -4mm\evensidemargin\oddsidemargin
%\textwidth 170mm\textheight 257mm\columnsep 7mm
\setlength{\fboxrule}{0.2ex}
\setlength{\fboxsep}{0.6ex}

\pagestyle{empty}
\begin{document}
\small{v.17.1}
\hfill\small{2017/6/2 実施}
\begin{center}
{\gt\large{情報科学科 数式処理実習ペア試験問題}}
\end{center}
\vspace{5mm}

全部で3問($\times 2$).資料を参考にして以下の問題をMapleで解き,出力して提出せよ.60点以上が合格.何番をやっているかが分かるようにせよ.

\begin{enumerate}
\item 
\begin{enumerate}
\item パラメトリックプロットと微分(15点) 

資料を参考にして次の関係を満たすグラフを$t=-2\pi..2\pi$でプロットせよ.さらに,$\displaystyle \frac{dy}{dx}$を求めよ.結果は$t$の関数のままでよい\footnote{寺田・坂田,「演習と応用 微分積分」(サイエンス社,2003),p.19, 問題10.1(2).}.
\begin{eqnarray*}
x &=& \cos(t) + t \, \sin(t)\\
y &=& \sin(t) - t \, \cos(t)
\end{eqnarray*}

\item 直交関数系の積分(15点)

以下の関数$f_{1,1}, f_{2,3}$を$x=-\pi..\pi$でプロットし,さらにその範囲で積分せよ.
\begin{eqnarray*}
f_{1,1} &=& \sin x \, \sin x\\
f_{2,3} &=& \sin 2x \, \sin 3x
\end{eqnarray*}
さらに,以上の結果を一般化すると
\begin{equation*}
F_{n,m} = \int_{-\pi}^{\pi} \sin nx \, \sin mx \, {\textrm d}x =
\left\{ \begin{array}{cc}
\pi \, (n=m) \\
0 \, (n\ne m)
\end{array}
\right.
\end{equation*}
が成立する.その理由をプロットから定性的に説明せよ.ただし,$n=m$で定量的に$\pi$になることの説明を解として求めているのではない.
\end{enumerate}
\item
\begin{enumerate}
\item 同時対角化(15点)

実対称行列$\displaystyle \boldsymbol{A}= \left( \begin {array}{ccc} 
1&0&-2\\ 
0&2&0 \\ 
-2&0&1
\end {array} \right)$,
$\displaystyle \boldsymbol{B}= \left( \begin {array}{ccc} 
3&0&2\\ 
0&-3&0 \\ 
2&0&3
\end {array} \right)$について,
\begin{enumerate}
\item $\boldsymbol{AB}=\boldsymbol{BA}$を確かめよ.
\item $\boldsymbol{A},\boldsymbol{B}$を直交行列$\boldsymbol{P}$によって同時に対角化せよ\footnote{寺田・木村,「演習と応用 線形代数」(サイエンス社,2005),p.117, 例題3,問題3.1}.
\end{enumerate}
\item 2次形式(15点)

2次形式
\begin{equation*}
f=3x_1^2+2x_2^2+4x_3^2+4x_1x_2+4x_1x_3
\end{equation*}
を行列
$\displaystyle \boldsymbol{P}= \left( \begin {array}{ccc} 
-\frac{2}{3}&-\frac{1}{3}&\frac{2}{3} \\ 
\frac{2}{3}&-\frac{2}{3}&\frac{1}{3} \\ 
\frac{1}{3}&\frac{2}{3}&\frac{2}{3} \\ 
\end {array} \right)$
によって変数変換せよ
\footnote{寺田・木村,「演習と応用 線形代数」(サイエンス社,2005),p.119, 例題4改, p.120 5.1(a)}.
\end{enumerate}

\pagebreak
\item
\begin{enumerate}
\item (10点)
定数$a,b,c$は,$\displaystyle a+b+c=1,\, ab+bc+ca=-2, \,abc=-1$を満たすとする.
\begin{enumerate}
\item
$\displaystyle a^2+b^2+c^2=\fbox{ ア }$\,,\, $\displaystyle \frac{1}{a}+\frac{1}{b}+\frac{1}{c}=\fbox{ イ }$である.

次に,$\displaystyle \frac{1}{a}+\frac{1}{b}+\frac{1}{c}=\fbox{ イ }$の両辺を2乗することで
\begin{equation*}
\displaystyle \frac{1}{a^2}+\frac{1}{b^2}+\frac{1}{c^2}=\fbox{ ウ }
\end{equation*}
であることがわかる.

\item
$x$の2次式$A$を
\begin{equation*}
A=\left(ax-\frac{1}{a} \right)^2+
\left(bx-\frac{1}{b} \right)^2+
\left(cx-\frac{1}{c} \right)^2
\end{equation*}
とおく.
\begin{equation*}
A = \fbox{ エ }x^2 -\fbox{ オ }x+\fbox{ カ }
\end{equation*}
であり,$A=7$を満たす$x$の値は
$\displaystyle 
\frac{{\fbox{ キ }} \pm \sqrt{\fbox{ クケ }}}{\fbox{ コ }}
$
である\footnote{2014年度大学入試センター試験数学I・A追試験第1問}. 
\end{enumerate}
\item (30点)
問3-(a)において,$\displaystyle a+b+c=1.1,\, ab+bc+ca=-2.2, \,abc=-1.1$を満たすと読み替えて,\fbox{ ア }から\fbox{ コ }を求めよ.ただし数値を変えたので,\fbox{ ア }等には箱にこだわらず,小数が入る.さらに,$\displaystyle 
\frac{{\fbox{ キ }} \pm \sqrt{\fbox{ クケ }}}{\fbox{ コ }}
$は浮動小数点数2個となる.
\end{enumerate}

\end{enumerate}
\end{document}