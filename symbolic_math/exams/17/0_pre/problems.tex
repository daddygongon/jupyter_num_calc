\documentclass[12pt,a4j]{jarticle}
\usepackage[dvips]{graphicx}
\usepackage{amsmath,amsthm,amssymb}
%\topmargin -15mm\oddsidemargin -4mm\evensidemargin\oddsidemargin
%\textwidth 170mm\textheight 257mm\columnsep 7mm
\setlength{\fboxrule}{0.2ex}
\setlength{\fboxsep}{0.6ex}

\pagestyle{empty}
\begin{document}
\small{v17.0  = v.16.2} 
\hfill\small{2017/05/19 pairリハーサル試験として実施,2016/07/01 実施}
\begin{center}
{\gt\large{情報科学科 数式処理演習 pairリハーサル試験 試験問題}}
\end{center}
\vspace{5mm}

以下の問題をMapleを用いて自力で解き,出力して提出せよ.60 点以下ならチーム解消.

\begin{enumerate}

\item 
\begin{enumerate}
\item 
(Einstein結晶のエネルギー) 次の関数$E(x)$を求めてx=0..2でプロットせよ.(15点)
\begin{align*}
Z(x) = \frac{\exp(1/x)}{1-\exp(-1/x)} \\
E(x) = x^2 \frac{\rm d}{{\rm d}x} \log \left(Z(x)\right)
\end{align*}

\item
資料を参考にして,次の2重積分を求めよ.(15点)
\begin{equation*}
\int \int_D \sqrt{2x^2-y^2}dxdy,\hspace{5mm} D:0\leqq y \leqq x \leqq 1 
\end{equation*}

\end{enumerate}

\item 
\begin{enumerate}
\item 
行列$\displaystyle A= 
\left[ \begin {array}{ccc} 
1&1&3\\ 
\noalign{\medskip}-1&0&1\\ 
\noalign{\medskip}1&2&1
\end {array} \right] $
の対角化行列を求めて,対角化せよ.(15点)
\item
資料を参考にして,行列$\displaystyle  
\left[ \begin {array}{cc} 
1/\sqrt{2}&a\\ 
\noalign{\medskip}b&-1/\sqrt{2}
\end {array} \right] $が直交行列であるとき,$a,b$を求めよ.(15点)

\end{enumerate}


\item 
$p$を実数とし,$f(x)=x^3-p\,\, x$とする.

\begin{enumerate}
\item
関数$f(x)$が極値をもつための$p$の条件を求めよう.$f(x)$の導関数は,
\begin{equation*}
f'(x) = \fbox{ ア }\,\,  x^{\,\,\fbox{ イ }}-p
\end{equation*}
である.したがって,$f(x)$が$x=a$で極値をとるならば,
\begin{equation*}
\fbox{ ア }\,\,  a^{\,\,\fbox{ イ }}-p=\fbox{ ウ }
\end{equation*}
が成り立つ.さらに$x=a$の前後での$f'(x)$の符号の変化を考えることにより,$p$が条件\fbox{ エ $(p>0)$}を満たす場合は$f(x)$は必ず極値を持つことがわかる.

\item 
関数$f(x)$が$\displaystyle x=\frac{p}{3}$で極値をとるとする.また,曲線$y=f(x)$を$C$とし,$C$上の点$\displaystyle \left(\frac{p}{3}, f\left(\frac{p}{3}\right) \right)$をAとする.

$f(x)$が$\displaystyle x=\frac{p}{3}$で極値をとることから,$p=\fbox{ オ }$であり,$f(x)$は$x=\fbox{ カキ }$で極大値をとり,$x=\fbox{ ク }$で極小値をとる.

曲線$C$の接線で,点Aを通り傾きが0でないものを$l$とする.$l$の方程式を求めよう.$l$と$C$の接点の$x$座標を$b$とすると,$l$は点$(b, f(b))$における$C$の接線であるから,$l$の方程式は$b$を用いて
\begin{equation*}
y= \left(\fbox{ ケ }\,\,  b^2 - \fbox{ コ }\right)(x-b)+f(b)
\end{equation*}
と表すことができる.また,$l$は点Aを通るから,方程式
\begin{equation*}
\fbox{ サ }\,\, b^3-\fbox{ シ }\,\, b^2+1=0
\end{equation*}
を得る.この方程式を解くと,
\begin{equation*}
b = \fbox{ ス }\,\,, \frac{\fbox{ セソ }}{\fbox{ タ }}
\end{equation*}
であるが,$l$の傾きが0でないことから,$l$の方程式は
\begin{equation*}
y = \frac{\fbox{ チツ }}{\fbox{ テ }}\,\, x+\frac{\fbox{ ト }}{\fbox{ ナ }}
\end{equation*}
である.

点Aを頂点とし,原点を通る放物線を$D$とする.$l$と$D$で囲まれた図形のうち,不等式$x \geqq 0$の表す領域に含まれる部分の面積$S$を求めよう.$D$の方程式は,
\begin{equation*}
y = \fbox{ ニ }\,\, x^2 -\fbox{ ヌ }\,\, x
\end{equation*}
であるから,定積分を計算することにより,$\displaystyle S=\frac{\fbox{ ネノ }}{24}$となる.(10点)

(2014年度大学入試センター試験 本試験 数学II・B第2問)
\end{enumerate}

\item.前問3(b)の$C$上の頂点Aの座標を$\displaystyle \left(\frac{p}{4}, f\left(\frac{p}{4}\right) \right)$と変えて問題を解け.ただし数値を変えたので,それほど複雑な数字にはならないが,\fbox{ オ },\fbox{ カキ }等には箱にこだわらず数字がはいる.最後は$\displaystyle S=\frac{34}{27}$ではなく,$\displaystyle S=\frac{352}{243}$になる.(30点)

\end{enumerate}


\end{document}