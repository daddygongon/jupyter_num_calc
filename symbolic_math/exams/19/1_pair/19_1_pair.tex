\documentclass[12pt,a4j]{jarticle}
\usepackage[dvips]{graphicx}
\usepackage{amsmath,amsthm,amssymb}
%\topmargin -15mm\oddsidemargin -4mm\evensidemargin\oddsidemargin
%\textwidth 170mm\textheight 257mm\columnsep 7mm
\setlength{\fboxrule}{0.2ex}
\setlength{\fboxsep}{0.6ex}

%%%%% 問題番号 %%%%%%%%%%%%%%%%%%%%%%%%%%%%
\def\labelenumi{[~\theenumi~]}  % 大問[1][2],... と付番したい場合
%\def\labelenumi{}              % 大問は1個で付番なしの場合
\def\labelenumii{(\arabic{enumii})}
%\def\labelenumi{(\arabic{enumi})}
%%%%%%%%%%%%%%%%%%%%%%%%%%%%%%%%%%%%%%%%%%

\pagestyle{empty}
\begin{document}
\small{v.19.1} 
\hfill\small{2019/06/14 実施}
\begin{center}
{\gt\large{情報科学科 数式処理演習 ペア 試験問題}}
\end{center}
\vspace{5mm}

以下の問題をpythonを用いてペアで解き,出力して提出せよ.60 点以下でグループ解散.

\begin{enumerate}

\item 
\begin{enumerate}
\item
  (偏微分,saddle point)
  関数$f(x,y)=x^2 - y^2$の2次偏導関数$f_{xx}, f_{xy}, f_{yx}, f_{yy}$を求めよ.
  また,$(x,y)=(0,0)$での判別式
  \begin{equation*}
    D = {f_{xy}(0,0)}^2-f_{xx}(0,0)f_{yy}(0,0) > 0
  \end{equation*}
  を確かめよ.さらに,関数$f(x,y)$をplot3dして鞍点(saddle point)の意味を確認せよ.
  (15点)
  
\item 
  (フーリエ積分)
  関数$f(x)= \sin(x) \sin(2x)$の不定積分を求めよ.$f(x)$を$x=-\pi..\pi$でプロットし,この区間での積分値を求めよ.結果についてコメントせよ.
  (15点)
\end{enumerate}

\item 
\begin{enumerate}
\item 
(ヌルスペース)
行列$A = \left(\begin{array}{cccc}
4 & -1 & -1 & 1\\
1 & 2 & -1 & -2\\
\end{array}
\right)
$を表現行列とする$R^4\rightarrow R^2$の線形写像$f$のKer($f$)の1組の基底を求めよ.(15点)

\item
(対角化)
行列$A = \left(\begin{array}{cccc}
4 & -1 & -1 \\
1 & 2 & -1 \\
3 & -1 & 0
\end{array}
\right)
$の固有値と固有ベクトルを求めよ.
また,対角化行列$P$を求めて,
$P^{-1}AP$と$P^{t}AP$を求め,違いを確かめよ.(15点)
\end{enumerate}


\item (2015年度大学入試センター試験 本試験 数学II・B第2問)

以下のセンター試験問題をpythonでcode化せよ.ただし,関数$f(x)$は
\begin{verbatim}
f =Rational(1,2)*x**2
f.subs({x:a})
\end{verbatim}
などとするべし.

  \begin{enumerate}
  \item
    関数$f(x)=\frac{1}{2}x^2$の$x=a$における微分係数$f'(a)$を
    求めよう.$h$が0でないとき,$x$が$a$から$a+h$まで変化するときの$f(x)$の
    平均変化率は$\fbox{ ア }+\frac{h}{\fbox{ イ }}$である.
    したがって,求める微分係数は
    \begin{equation*}
      f'(a) = \lim_{h \rightarrow \fbox{ ウ }}
      \left(\fbox{ ア }+\frac{h}{\fbox{ イ }}\right) = \fbox{ エ }
    \end{equation*}
    である.
  \item
    放物線$y=\frac{1}{2}x^2$を$C$とし,$C$上に点P$\left(a, \frac{1}{2}a^2\right)$をとる.
    ただし,$a > 0$とする.点Pにおける$C$の接線$l$の方程式は
    \begin{equation*}
      y = \fbox{ オ }x - \frac{1}{\fbox{ カ }}a^2
    \end{equation*}
    である.
    直線$l$と$x$軸との交点$Q$の座標は$\left(\frac{\fbox{ キ }}{\fbox{ ク }}, 0\right)$である.
    点Qを通り$l$に垂直な直線を$m$とすると,$m$の方程式は
    \begin{equation*}
      y = \frac{\fbox{ ケコ }}{\fbox{ サ }}x + \frac{\fbox{ シ }}{\fbox{ ス }}
    \end{equation*}
    である.
  \item
    直線$m$と$y$軸との交点をAとする.三角形APQの面積を$S$とおくと
    \begin{equation*}
      S = \frac{a\left(a^2+\fbox{ セ }\right)}{\fbox{ ソ }}
    \end{equation*}
    となる.また,$y$軸と線分APおよび曲線$C$によって囲まれた図形の面積を$T$とおくと
    \begin{equation*}
      T = \frac{a\left(a^2+\fbox{ タ }\right)}{\fbox{ チツ }}
    \end{equation*}
    となる.

    $a>0$の範囲における$S-T$の値について調べよう.
    \begin{equation*}
      S -T = \frac{a\left(a^2-\fbox{ テ }\right)}{\fbox{ トナ }}
    \end{equation*}
    である.
    $a>0$であるから,$S-T>0$となるような$a$ のとり得る値の範囲は
    $ a > \sqrt{\fbox{ ニ }}$である.
    また,$a>0$のときの$S-T$の増減を調べると,
    $S-T$は$a=\fbox{ ヌ }$で最小値$\frac{\fbox{ ネノ }}{\fbox{ ハヒ }}$をとることがわかる.

    (10点)
  \end{enumerate}

\item 前問の関数を$f(x)=0.49x^2$および放物線$C$の方程式を$y=0.49x^2$として問題を解け.
  数値解となるので,答えはかっこによらず小数点となる.最後の最小値は-0.08676940ぐらい.(30点)


\end{enumerate}


\end{document}
