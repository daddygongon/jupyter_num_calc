\documentclass[12pt,a4j]{jarticle}
\usepackage[dvips]{graphicx,color}
\usepackage{amsmath,amsthm,amssymb}
%\topmargin -15mm\oddsidemargin -4mm\evensidemargin\oddsidemargin
%\textwidth 170mm\textheight 257mm\columnsep 7mm
\setlength{\fboxrule}{0.2ex}
\setlength{\fboxsep}{0.6ex}

\pagestyle{empty}
\begin{document}
\small{v.18.1}
\hfill\small{2018/6/8 実施}
\begin{center}
{\gt\large{情報科学科 数式処理実習ペア試験問題}}
\end{center}
\vspace{5mm}

全部で4問.資料を参考にして以下の問題をpythonで解き,グループごとに1部出力して提出せよ.全員の名前を忘れんように.60点以上が合格.何番をやっているかが分かるようにせよ.

\begin{enumerate}

\item 
\begin{enumerate}
\item 微分(15点) 

次の関数を微分せよ.
\begin{eqnarray*}
\sin^{-1}{\left (\frac{x^{2} - 1}{x^{2} + 1} \right )}
\end{eqnarray*}
ただし,\verb|x = symbols('x',positive = True)|で
変数の値を制約しておくと,式がとても簡単になる.

\item 2重積分(15点)
次の2重積分を資料を参考にして求めよ.
\begin{equation*}
\int \int_D \sqrt{x^2+4y^2}dxdy, \, \, 
D:0\leqq x \leqq y \leqq 1
\end{equation*}
\end{enumerate}

\item
\begin{enumerate}
\item 写像のIm, Ker(15点)

次の変換の行列$A$で表される写像$f$のIm, Kerのそれぞれの次元と1組の基底を求めなさい.
\begin{equation*}
A = \left(\begin{matrix}1 & 1 & 3 & 3\\0 & 1 & 1 & 2\\1 & 0 & 2 & 1\end{matrix}\right)
\end{equation*}
\item  行列の対角化(15点)

次の行列$A$を対角化する変換の行列$P$を求めて,対角化しなさい.
\begin{equation*}
A=\left(\begin{matrix}0.5 & 0 & 0.5\\0 & 0.5 & 0.5\\0.5 & 0.5 & 0\end{matrix}\right)
\end{equation*}

\end{enumerate}

\pagebreak
\item
\begin{enumerate}
\item (10点)
Oを原点とする座標平面上の放物線$y=x^2+1$を$C$とし,
点$(a,2a)$をPとする.

点Pを通り,放物線$C$に接する直線の方程式を求めよう.

$C$上の点$(t, t^2+1)$における接線の方程式は
\begin{equation*}
y=\fbox{ ア }\,\, tx -t^2 +\fbox{ イ }
\end{equation*}
である.
この直線がPを通るとすると,$t$は方程式
\begin{equation*}
t^2 - \fbox{ ウ }\,\,at + \fbox{ エ }\,a -\fbox{ オ }\, = 0
\end{equation*}
を満たすから,
$\displaystyle 
t = {\fbox{ カ }}\,\,a - \fbox{ キ },\,\,
{\fbox{ ク }}$
である.よって,
$\displaystyle a \neq {\fbox{ ケ }}$
のとき,Pを通る$C$の接線は2本あり,それらの方程式は,
\begin{equation}
y = \left( {\fbox{ コ }}\,\,a -{\fbox{ サ }}\,\right)\,x -
{\fbox{ シ }}\,\,a^2 + {\fbox{ ス }}\, a
\end{equation}
と
\begin{equation*}
y = \fbox{ セ }\,\,x
\end{equation*}
ある\footnote{2017年度大学入試センター試験数学II・B本試験第2問(a)}. 

\item (20点)
問3-(a)において,放物線$C$を$\displaystyle y=x^2 +2 $, 点Pを$(a, \sqrt{8}a)$と読み替えて,$t$の値を求めよ.

\end{enumerate}

\item 問3-(a)の続き\footnote{2017年度大学入試センター試験数学II・B本試験第2問(b,c)}(10点)

(a)の方程式(1)で表される直線を$l$とする.$l$と$y$軸との
交点をR$(0,r)$とすると,
$\displaystyle r = -{\fbox{ シ }}\,\,a^2 + {\fbox{ ス }}\, a$
である.
$\displaystyle r > 0$
となるのは,
$\displaystyle {\fbox{ ソ }}\, < a < {\fbox{ タ }} $
のときであり,このとき,三角形OPRの面積$S$は
\begin{equation*}
S = {\fbox{ チ }} \left(a^{\fbox{ツ}}- a^{\fbox{テ}} \right)
\end{equation*}
となる.

$\displaystyle {\fbox{ ソ }}\, < a < {\fbox{ タ }} $
のとき,$S$の増減を調べると,$S$は
$\displaystyle a = \frac{{\fbox{ ト }}}{{\fbox{ ナ }}}$
で最大値
$\displaystyle \frac{{\fbox{ ニ }}}{{\fbox{ ヌネ }}}$
をとることがわかる.

$\displaystyle {\fbox{ ソ }}\, < a < {\fbox{ タ }} $
のとき,放物線$C$と直線$l$および2直線
$x=0, x=a$で囲まれた図形の面積を$T$とすると
\begin{equation*}
T = \frac{{\fbox{ ノ }}}{{\fbox{ ハ }}}\,\, a^3
- {\fbox{ ヒ }}\,\,a^2
+ {\fbox{ フ }}
\end{equation*}
である.
$\displaystyle \frac{2}{3} \leqq a < 1$
の範囲において$T$は単調増加する.


\end{enumerate}
\end{document}